\documentclass[a4paper,10pt,fleqn]{article}
\title{{\Huge Capstone Project}}%%% \\Machine Learning Engineer Nanodegree}

\author{Ryosuke Honda}
\date{\today}
\begin{document}
\maketitle

\section{Definition}
\subsection{Project Overview}
Student provides a high-level overview of the project in layman’s terms. Background information such as the problem domain, the project origin, and related data sets or input data is given.


\subsection{Problem Statement}
The problem which needs to be solved is clearly defined. A strategy for solving the problem, including discussion of the expected solution, has been made.







{\Large The CIFAR-10 dataset}\\
The CIFAR-10 dataset consists of 60000 32x32x3(width=height=32 and RGB=3) color images in 10 classes, with 6000 images. 50000 images are the training images and 10000 images are the test images.\\
The classes are completely mutually exclusive. There is no overlap within each image class.

In this task, I'll discuss the image recognition by Convolutional Neural Network(CNN).





\subsection{Metrics}
Metrics used to measure performance of a model or result are clearly defined. Metrics aper classre justified based on the characteristics of the problem.


\section{Analysis}
\subsection{Data Exploration}
If a dataset is present, features and calculated statistics relevant to the problem have been reported and discussed, along with a sampling of the data. In lieu of a dataset, a thorough description of the input space or input data has been made. Abnormalities or characteristics about the data or input that need to be addressed have been identified.


\subsection{Exploratory Visualization}
A visualization has been provided that summarizes or extracts a relevant characteristic or feature about the dataset or input data with thorough discussion. Visual cues are clearly defined.


\subsection{Algorithms and Techniques}
Algorithms and techniques used in the project are thoroughly discussed and properly justified based on the characteristics of the problem.


\subsection{Benchmark}
Student clearly defines a benchmark result or threshold for comparing performances of solutions obtained.


\section{Methodology}
\subsection{Data Preprocessing}
All preprocessing steps have been clearly documented. Abnormalities or characteristics about the data or input that needed to be addressed have been corrected. If no data preprocessing is necessary, it has been clearly justified.


\subsection{Implementation}
The process for which metrics, algorithms, and techniques were implemented with the given datasets or input data has been thoroughly documented. Complications that occurred during the coding process are discussed.

\subsection{Refinement}
The process of improving upon the algorithms and techniques used is clearly documented. Both the initial and final solutions are reported, along with intermediate solutions, if necessary.



\section{Result}
\subsection{Model Evaluation and Validation}
The final model’s qualities — such as parameters — are evaluated in detail. Some type of analysis is used to validate the robustness of the model’s solution.

\subsection{Justification}
The final results are compared to the benchmark result or threshold with some type of statistical analysis. Justification is made as to whether the final model and solution is significant enough to have adequately solved the problem.




\section{Conclusion}
\subsection{Free-Form Visualization}
A visualization has been provided that emphasizes an important quality about the project with thorough discussion. Visual cues are clearly defined.

\subsection{Reflection}
Student adequately summarizes the end-to-end problem solution and discusses one or two particular aspects of the project they found interesting or difficult.

\subsection{Improvement}
Discussion is made as to how one aspect of the implementation could be improved. Potential solutions resulting from these improvements are considered and compared/contrasted to the current solution.





\end{document}