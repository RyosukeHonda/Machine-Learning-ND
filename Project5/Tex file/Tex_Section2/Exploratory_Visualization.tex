\subsection{Exploratory Visualization}
A visualization has been provided that summarizes or extracts a relevant characteristic or feature about the dataset or input data with thorough discussion. Visual cues are clearly defined.


There are 60,000 of images in total. Figure.\ref{fig:two} (50,000 images for training data and 10,000 images for test data) shows the samples of the images. I plotted 10 images for each class.Figure.\ref{fig:three} and Figure.\ref{fig:four} shows the distribution of the data. For the training dataset, each class has 5,000 images and for the test dataset, each class has 1,000 images.
The label 0 to 9 corresponds to 'Airplane','Automobile','Bird','Cat','Deer','Dog','Frog','Horse','Ship, and 'Truck'.
\begin{figure}[htbp]

\begin{center}
\includegraphics[width=10cm]{picture/random_sample.png}
\end{center}
\caption{Sample of the Images}
\label{fig:two}

\end{figure}

\begin{figure}[h]
\begin{minipage}{0.5\hsize}
	\begin{center}
	\includegraphics[width=5cm]{picture/Distribution_of_Training_Data.png}
	\end{center}
	\caption{Distribution of Training Data}
	\label{fig:three}
\end{minipage}
\begin{minipage}{0.5\hsize}
\begin{center}
\includegraphics[width=5cm]{picture/Distribution_of_Test_Data.png}
\end{center}
 \caption{Distribution of Test Data}
  \label{fig:four}
 \end{minipage}
\end{figure}

